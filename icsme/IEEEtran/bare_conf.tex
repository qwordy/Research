
%% bare_conf.tex
%% V1.4b
%% 2015/08/26
%% by Michael Shell
%% See:
%% http://www.michaelshell.org/
%% for current contact information.
%%
%% This is a skeleton file demonstrating the use of IEEEtran.cls
%% (requires IEEEtran.cls version 1.8b or later) with an IEEE
%% conference paper.
%%
%% Support sites:
%% http://www.michaelshell.org/tex/ieeetran/
%% http://www.ctan.org/pkg/ieeetran
%% and
%% http://www.ieee.org/

%%*************************************************************************
%% Legal Notice:
%% This code is offered as-is without any warranty either expressed or
%% implied; without even the implied warranty of MERCHANTABILITY or
%% FITNESS FOR A PARTICULAR PURPOSE! 
%% User assumes all risk.
%% In no event shall the IEEE or any contributor to this code be liable for
%% any damages or losses, including, but not limited to, incidental,
%% consequential, or any other damages, resulting from the use or misuse
%% of any information contained here.
%%
%% All comments are the opinions of their respective authors and are not
%% necessarily endorsed by the IEEE.
%%
%% This work is distributed under the LaTeX Project Public License (LPPL)
%% ( http://www.latex-project.org/ ) version 1.3, and may be freely used,
%% distributed and modified. A copy of the LPPL, version 1.3, is included
%% in the base LaTeX documentation of all distributions of LaTeX released
%% 2003/12/01 or later.
%% Retain all contribution notices and credits.
%% ** Modified files should be clearly indicated as such, including  **
%% ** renaming them and changing author support contact information. **
%%*************************************************************************


% *** Authors should verify (and, if needed, correct) their LaTeX system  ***
% *** with the testflow diagnostic prior to trusting their LaTeX platform ***
% *** with production work. The IEEE's font choices and paper sizes can   ***
% *** trigger bugs that do not appear when using other class files.       ***                          ***
% The testflow support page is at:
% http://www.michaelshell.org/tex/testflow/



\documentclass[conference]{IEEEtran}
% Some Computer Society conferences also require the compsoc mode option,
% but others use the standard conference format.
%
% If IEEEtran.cls has not been installed into the LaTeX system files,
% manually specify the path to it like:
% \documentclass[conference]{../sty/IEEEtran}





% Some very useful LaTeX packages include:
% (uncomment the ones you want to load)


% *** MISC UTILITY PACKAGES ***
%
%\usepackage{ifpdf}
% Heiko Oberdiek's ifpdf.sty is very useful if you need conditional
% compilation based on whether the output is pdf or dvi.
% usage:
% \ifpdf
%   % pdf code
% \else
%   % dvi code
% \fi
% The latest version of ifpdf.sty can be obtained from:
% http://www.ctan.org/pkg/ifpdf
% Also, note that IEEEtran.cls V1.7 and later provides a builtin
% \ifCLASSINFOpdf conditional that works the same way.
% When switching from latex to pdflatex and vice-versa, the compiler may
% have to be run twice to clear warning/error messages.






% *** CITATION PACKAGES ***
%
%\usepackage{cite}
% cite.sty was written by Donald Arseneau
% V1.6 and later of IEEEtran pre-defines the format of the cite.sty package
% \cite{} output to follow that of the IEEE. Loading the cite package will
% result in citation numbers being automatically sorted and properly
% "compressed/ranged". e.g., [1], [9], [2], [7], [5], [6] without using
% cite.sty will become [1], [2], [5]--[7], [9] using cite.sty. cite.sty's
% \cite will automatically add leading space, if needed. Use cite.sty's
% noadjust option (cite.sty V3.8 and later) if you want to turn this off
% such as if a citation ever needs to be enclosed in parenthesis.
% cite.sty is already installed on most LaTeX systems. Be sure and use
% version 5.0 (2009-03-20) and later if using hyperref.sty.
% The latest version can be obtained at:
% http://www.ctan.org/pkg/cite
% The documentation is contained in the cite.sty file itself.






% *** GRAPHICS RELATED PACKAGES ***
%
\ifCLASSINFOpdf
  % \usepackage[pdftex]{graphicx}
  % declare the path(s) where your graphic files are
  % \graphicspath{{../pdf/}{../jpeg/}}
  % and their extensions so you won't have to specify these with
  % every instance of \includegraphics
  % \DeclareGraphicsExtensions{.pdf,.jpeg,.png}
\else
  % or other class option (dvipsone, dvipdf, if not using dvips). graphicx
  % will default to the driver specified in the system graphics.cfg if no
  % driver is specified.
  % \usepackage[dvips]{graphicx}
  % declare the path(s) where your graphic files are
  % \graphicspath{{../eps/}}
  % and their extensions so you won't have to specify these with
  % every instance of \includegraphics
  % \DeclareGraphicsExtensions{.eps}
\fi
% graphicx was written by David Carlisle and Sebastian Rahtz. It is
% required if you want graphics, photos, etc. graphicx.sty is already
% installed on most LaTeX systems. The latest version and documentation
% can be obtained at: 
% http://www.ctan.org/pkg/graphicx
% Another good source of documentation is "Using Imported Graphics in
% LaTeX2e" by Keith Reckdahl which can be found at:
% http://www.ctan.org/pkg/epslatex
%
% latex, and pdflatex in dvi mode, support graphics in encapsulated
% postscript (.eps) format. pdflatex in pdf mode supports graphics
% in .pdf, .jpeg, .png and .mps (metapost) formats. Users should ensure
% that all non-photo figures use a vector format (.eps, .pdf, .mps) and
% not a bitmapped formats (.jpeg, .png). The IEEE frowns on bitmapped formats
% which can result in "jaggedy"/blurry rendering of lines and letters as
% well as large increases in file sizes.
%
% You can find documentation about the pdfTeX application at:
% http://www.tug.org/applications/pdftex





% *** MATH PACKAGES ***
%
%\usepackage{amsmath}
% A popular package from the American Mathematical Society that provides
% many useful and powerful commands for dealing with mathematics.
%
% Note that the amsmath package sets \interdisplaylinepenalty to 10000
% thus preventing page breaks from occurring within multiline equations. Use:
%\interdisplaylinepenalty=2500
% after loading amsmath to restore such page breaks as IEEEtran.cls normally
% does. amsmath.sty is already installed on most LaTeX systems. The latest
% version and documentation can be obtained at:
% http://www.ctan.org/pkg/amsmath





% *** SPECIALIZED LIST PACKAGES ***
%
%\usepackage{algorithmic}
% algorithmic.sty was written by Peter Williams and Rogerio Brito.
% This package provides an algorithmic environment fo describing algorithms.
% You can use the algorithmic environment in-text or within a figure
% environment to provide for a floating algorithm. Do NOT use the algorithm
% floating environment provided by algorithm.sty (by the same authors) or
% algorithm2e.sty (by Christophe Fiorio) as the IEEE does not use dedicated
% algorithm float types and packages that provide these will not provide
% correct IEEE style captions. The latest version and documentation of
% algorithmic.sty can be obtained at:
% http://www.ctan.org/pkg/algorithms
% Also of interest may be the (relatively newer and more customizable)
% algorithmicx.sty package by Szasz Janos:
% http://www.ctan.org/pkg/algorithmicx




% *** ALIGNMENT PACKAGES ***
%
%\usepackage{array}
% Frank Mittelbach's and David Carlisle's array.sty patches and improves
% the standard LaTeX2e array and tabular environments to provide better
% appearance and additional user controls. As the default LaTeX2e table
% generation code is lacking to the point of almost being broken with
% respect to the quality of the end results, all users are strongly
% advised to use an enhanced (at the very least that provided by array.sty)
% set of table tools. array.sty is already installed on most systems. The
% latest version and documentation can be obtained at:
% http://www.ctan.org/pkg/array


% IEEEtran contains the IEEEeqnarray family of commands that can be used to
% generate multiline equations as well as matrices, tables, etc., of high
% quality.




% *** SUBFIGURE PACKAGES ***
%\ifCLASSOPTIONcompsoc
%  \usepackage[caption=false,font=normalsize,labelfont=sf,textfont=sf]{subfig}
%\else
%  \usepackage[caption=false,font=footnotesize]{subfig}
%\fi
% subfig.sty, written by Steven Douglas Cochran, is the modern replacement
% for subfigure.sty, the latter of which is no longer maintained and is
% incompatible with some LaTeX packages including fixltx2e. However,
% subfig.sty requires and automatically loads Axel Sommerfeldt's caption.sty
% which will override IEEEtran.cls' handling of captions and this will result
% in non-IEEE style figure/table captions. To prevent this problem, be sure
% and invoke subfig.sty's "caption=false" package option (available since
% subfig.sty version 1.3, 2005/06/28) as this is will preserve IEEEtran.cls
% handling of captions.
% Note that the Computer Society format requires a larger sans serif font
% than the serif footnote size font used in traditional IEEE formatting
% and thus the need to invoke different subfig.sty package options depending
% on whether compsoc mode has been enabled.
%
% The latest version and documentation of subfig.sty can be obtained at:
% http://www.ctan.org/pkg/subfig




% *** FLOAT PACKAGES ***
%
%\usepackage{fixltx2e}
% fixltx2e, the successor to the earlier fix2col.sty, was written by
% Frank Mittelbach and David Carlisle. This package corrects a few problems
% in the LaTeX2e kernel, the most notable of which is that in current
% LaTeX2e releases, the ordering of single and double column floats is not
% guaranteed to be preserved. Thus, an unpatched LaTeX2e can allow a
% single column figure to be placed prior to an earlier double column
% figure.
% Be aware that LaTeX2e kernels dated 2015 and later have fixltx2e.sty's
% corrections already built into the system in which case a warning will
% be issued if an attempt is made to load fixltx2e.sty as it is no longer
% needed.
% The latest version and documentation can be found at:
% http://www.ctan.org/pkg/fixltx2e


%\usepackage{stfloats}
% stfloats.sty was written by Sigitas Tolusis. This package gives LaTeX2e
% the ability to do double column floats at the bottom of the page as well
% as the top. (e.g., "\begin{figure*}[!b]" is not normally possible in
% LaTeX2e). It also provides a command:
%\fnbelowfloat
% to enable the placement of footnotes below bottom floats (the standard
% LaTeX2e kernel puts them above bottom floats). This is an invasive package
% which rewrites many portions of the LaTeX2e float routines. It may not work
% with other packages that modify the LaTeX2e float routines. The latest
% version and documentation can be obtained at:
% http://www.ctan.org/pkg/stfloats
% Do not use the stfloats baselinefloat ability as the IEEE does not allow
% \baselineskip to stretch. Authors submitting work to the IEEE should note
% that the IEEE rarely uses double column equations and that authors should try
% to avoid such use. Do not be tempted to use the cuted.sty or midfloat.sty
% packages (also by Sigitas Tolusis) as the IEEE does not format its papers in
% such ways.
% Do not attempt to use stfloats with fixltx2e as they are incompatible.
% Instead, use Morten Hogholm'a dblfloatfix which combines the features
% of both fixltx2e and stfloats:
%
% \usepackage{dblfloatfix}
% The latest version can be found at:
% http://www.ctan.org/pkg/dblfloatfix




% *** PDF, URL AND HYPERLINK PACKAGES ***
%
%\usepackage{url}
% url.sty was written by Donald Arseneau. It provides better support for
% handling and breaking URLs. url.sty is already installed on most LaTeX
% systems. The latest version and documentation can be obtained at:
% http://www.ctan.org/pkg/url
% Basically, \url{my_url_here}.




% *** Do not adjust lengths that control margins, column widths, etc. ***
% *** Do not use packages that alter fonts (such as pslatex).         ***
% There should be no need to do such things with IEEEtran.cls V1.6 and later.
% (Unless specifically asked to do so by the journal or conference you plan
% to submit to, of course. )


% correct bad hyphenation here
\hyphenation{op-tical net-works semi-conduc-tor}


\begin{document}
%
% paper title
% Titles are generally capitalized except for words such as a, an, and, as,
% at, but, by, for, in, nor, of, on, or, the, to and up, which are usually
% not capitalized unless they are the first or last word of the title.
% Linebreaks \\ can be used within to get better formatting as desired.
% Do not put math or special symbols in the title.
\title{Bare Demo of IEEEtran.cls\\ for IEEE Conferences}


% author names and affiliations
% use a multiple column layout for up to three different
% affiliations
\author{\IEEEauthorblockN{Michael Shell}
\IEEEauthorblockA{School of Electrical and\\Computer Engineering\\
Georgia Institute of Technology\\
Atlanta, Georgia 30332--0250\\
Email: http://www.michaelshell.org/contact.html}
\and
\IEEEauthorblockN{Homer Simpson}
\IEEEauthorblockA{Twentieth Century Fox\\
Springfield, USA\\
Email: homer@thesimpsons.com}
\and
\IEEEauthorblockN{James Kirk\\ and Montgomery Scott}
\IEEEauthorblockA{Starfleet Academy\\
San Francisco, California 96678--2391\\
Telephone: (800) 555--1212\\
Fax: (888) 555--1212}}

% conference papers do not typically use \thanks and this command
% is locked out in conference mode. If really needed, such as for
% the acknowledgment of grants, issue a \IEEEoverridecommandlockouts
% after \documentclass

% for over three affiliations, or if they all won't fit within the width
% of the page, use this alternative format:
% 
%\author{\IEEEauthorblockN{Michael Shell\IEEEauthorrefmark{1},
%Homer Simpson\IEEEauthorrefmark{2},
%James Kirk\IEEEauthorrefmark{3}, 
%Montgomery Scott\IEEEauthorrefmark{3} and
%Eldon Tyrell\IEEEauthorrefmark{4}}
%\IEEEauthorblockA{\IEEEauthorrefmark{1}School of Electrical and Computer Engineering\\
%Georgia Institute of Technology,
%Atlanta, Georgia 30332--0250\\ Email: see http://www.michaelshell.org/contact.html}
%\IEEEauthorblockA{\IEEEauthorrefmark{2}Twentieth Century Fox, Springfield, USA\\
%Email: homer@thesimpsons.com}
%\IEEEauthorblockA{\IEEEauthorrefmark{3}Starfleet Academy, San Francisco, California 96678-2391\\
%Telephone: (800) 555--1212, Fax: (888) 555--1212}
%\IEEEauthorblockA{\IEEEauthorrefmark{4}Tyrell Inc., 123 Replicant Street, Los Angeles, California 90210--4321}}




% use for special paper notices
%\IEEEspecialpapernotice{(Invited Paper)}




% make the title area
\maketitle

% As a general rule, do not put math, special symbols or citations
% in the abstract
\begin{abstract}
The abstract goes here.
\end{abstract}

% no keywords




% For peer review papers, you can put extra information on the cover
% page as needed:
% \ifCLASSOPTIONpeerreview
% \begin{center} \bfseries EDICS Category: 3-BBND \end{center}
% \fi
%
% For peerreview papers, this IEEEtran command inserts a page break and
% creates the second title. It will be ignored for other modes.
\IEEEpeerreviewmaketitle



\section{Introduction}
Concurrent programs are pervasive in nowadays software development activities. Using concurrency rightly in programs can exploit the calculation ability better with the rapid development of multi-core system. However, concurrent programs are known hard to write correctly for multiple threads accessing objects simultaneously or depending on each other usually need complex synchronization and hard to debug for the uncertainty of thread interleaving which makes it difficult to reproduce the bug. Developers often struggle with various of synchronization methods and subtle concurrent bugs. There are much research about concurrent programming in the literature such as data race detection, atomicity violation detection or deadlock detection. Some empirical work also give us much help. Rui Gu \cite{conf/sigsoft/GuJSZL15} studied change history of thread synchronization. Gustavo Pinto \cite{journals/jss/PintoTFFB15} did a large-scale study on the usage of Java’s concurrent programming constructs.

Software projects evolves during years because of new functionalities, bugs, reorganization of code. A few of open source software platforms like github has been more and more popular in recent years. They hold a huge amount of software projects and their historic versions. Researchers have shown that software evolution history can provide much useful information for today's software development activities. Many studies focus on topics of software evolution such as refactoring, transformation patterns. Gustavo Santos \cite{conf/icsm/SantosAEDV15} studied system specific, source code transformations.

We studied concurrent programs from a perspective of software evolution history and found many change patterns about concurrent programming.

However, this work has to face several challenges:

1. The scale of open source software is increasing explosively as a result of some open source code platforms have become more and more popular. The change history of the open source software is also vast. Our interest is concurrent related commit, but they are hidden in the massive commit history. It requires much time and effort to identify whether a commit is concurrent related or not if doing it manually. We would like to adopt some automatic methods. Simple keyword matching algorithm will not work well because some commits just add or remove functionalities rather than modify original code.

2. The changes of code usually have complex relationship with the context not only in the file where change happens but also other files. Some change patterns have implicit dependency on the existing code. This raises a challenge to identify real change patterns which can be applied to other context correctly.

Our main contributions are:

1. 

2. We identify and classify change patterns in concurrent code and observe some interesting findings.

3. We give some inspirations to concurrent program or library developers and analysis tool developers.

The rest of paper is organized as follows: Section 2 presents the methodology of our study. Section 3 presents our result and discussion. Section 4 presents related work. Section 5 presents future work and Section 6 concludes.

\section{Methodology}
This section presents the project sources of our study, research questions and methods of doing the study. We have  developed a tool supporting the empirical study.

\subsection{Data set} We investigate 8 Java open-source projects from Github including Hadoop, Tomcat, Cassandra, Lucene-solr, Netty, Flink, Guava and Mahout as shown in Table 1. They are all popular, large-scale, active, representative Java open-source projects and cover different areas like distributed computing, web server, database, information retrieval, I/O and machine learning. The Hadoop project develops open-source software for reliable, scalable, distributed computing and has become one of the most famous Java open-source software for many years. Tomcat is the most popular implementation of the Java Servlet, JavaServer Pages, Java Expression Language and Java WebSocket technologies. Cassandra is a database system which can Manage massive amounts of data, fast, without losing sleep. Lucene-solr is two projects together in one respository in Github. Lucene is a search engine library and solr is a search engine server which uses lucene. Netty is an event-driven asynchronous network application framework. Flink is an open source stream processing framework with powerful stream- and batch-processing capabilities. Mahout is a machine learning project. Table 1 shows the lines of code in Java, the number of Java files and the number of commits of each project. All the projects are checked out for our study in December 2016.

\begin{table}
	\centering
	\caption{Projects information (LOC and \#Files are both of Java files)}
	\begin{tabular}{|c|c|c|c|}\hline
		Project&LOC&\#Files&\#Commits\\\hline
		Hadoop&1202764&7701&14930\\\hline
		Tomcat&301173&2192&17731\\\hline
		Cassandra&387980&2143&21982\\\hline
		Lucene-solr&918398&6310&26152\\\hline
		Netty&218131&2054&7759\\\hline
		Flink&414264&4068&9771\\\hline
		Guava&251205&1672&3850\\\hline
		Mahout&109584&1215&3703\\\hline
	\end{tabular}
\end{table}

\subsection{Research questions}
In order to understand the evolution of concurrent code better, we proposed 4 research questions:

\textbf{RQ1.} How many change patterns in concurrent programming?

Some change patterns are system-specific while some are global, which can be considered as knowledge. Developers made numerous commits to the project repository during software's whole life. Researchers have found some changes are similar , known as change patterns. We have a belief that there are also many change patterns in concurrent programming.

\textbf{RQ2.} How frequent do concurrent related code modification appear in different kinds of Java open-source projects?

Concurrent programming is very popular in today's Java development with the rapid developments of multi-core techniques which help exploit the power of concurrent programming. Java programming language provides convenient built-in concurrent libraries and users can also invoke third-party libraries. Although developers can use their own concurrent related classes or third-party libraries, they are always using the facilities provided by JDK by default. We want to know how frequent do concurrent related code modification in software projects. What are the differences of frequency in different kinds of software projects.

\textbf{RQ3.} What is the trend of concurrent programming construct usage statistically?

Java programming language offers many handy facilities for building concurrent programs. For example, language level constructs like synchronized and volatile are keywords of Java.  There are also API level constructs like notify method of an object and some concurrent related convenient classes such as the java.util.concurrent package. There always are more than one ways to finish a task in Java and the preferences of developers evolves fast. We are interested in  the trend of some common concurrent related constructs and the possible reasons hidden behind the phenomenon.

\textbf{RQ4.} How can these change patterns in history guide the development?

In order to better demonstrate these change patterns can really help developers understand concurrent programming practice, we are going to find the appropriate context in open-source projects and pull requests of  applying the change patterns. 

\subsection{Tool support}
We have developed a tool to collect and analyze data. The tool have the following functionalities.

\subsubsection{Collecting commits}
All the projects of our study are under git which is one of the most popular version control systems in the world. Some projects of the study used svn or some other version control systems before because they have long histories, but they all support git now. We employ JGit, a  lightweight, pure java library implementation of git, to retrieve all the commit logs in projects' histories. A typical commit log contains commit id which is a 20-character-long string uniquely identifying a commit, author, date and message. Once we get a commit id, we use "git show" command to show the log message and textual diff. The diff result contains one or more change files which contain one or more change hunks.

\subsubsection{Classification}
There are many commits which are not concurrent related in the commits which we have collected.  We need to select concurrent related commits. Yuan Tian et al. gave a successful example of identifing bug fixing patches using machine learning\cite{conf/icse/TianLL12}. We use machine learning to train and predict whether a commit is concurrent related. We adopt both text analysis and code analysis to extract features. A commit log uses natural language to present what was changed and why the change was made in most cases. We treat each commit log as a bag of words then match the words to a set of concurrent keywords which we have defined as the Java concurrent keywords like "synchronized", "volatile" and names of common classes or interfaces in Java libraries which are related to concurrency. We also do a code analysis based on the diff result. 12 features are extracted for each commit, which is shown in table.

We use the SVM\cite{journals/ml/CortesV95} algorithm to train and classify commits as concurrent-related or not. SVM is a supervised classification algorithm which needs both positive and negative labeled data for training. In our tool, we use an implementation of SVM, LIBSVM\cite{libsvm}. We manually label some data as a training data set first then train a model. The trained classifier selects 135 positive instances from all the commits which we have collected.

\section{Results}

\subsection{RQ1. How many change patterns in concurrent programming?}

The change patterns in concurrent programming can be divided into two categories: 

\subsection{RQ2. How frequent do concurrent related code modification appear in different kinds of Java open-source projects?}

We 

\subsection{RQ3. What is the trend of concurrent programming construct usage statistically?}

\subsection{RQ4. How can these change patterns in history guide the future development?}

\section{Discussion}
This research provides some implications from different kinds of perspectives.

(1) Developers are facing more and more concurrent programming requirements now. But concurrent programming is notoriously error-prone because of the complexity of data synchronization and thread interleaving. Our study gives developers some guidelines of writing concurrent programs. First, use handy concurrent libraries to finish the job instead of rewrite them by yourself unless all the available concurrent libraries cannot satisfy your requirement and you are absolutely confident of your comcurrent programming skills. Using existing libraries allows you to  write less code to finish the same work and enjoy the high quality of implementation which is always reliable, strong and fast. Second, always switch to new-version libraries because they usually provide higher performance and robustness.

(2) Automatic tools are needed to help developers inspect and revise concurrent programs with the help of history information. There has already been some tools, but they usually look for concurrent bugs such as race detection, deadlock detection and atomicity violation without considering software evolution history. Both project specific and project independent transformation patterns exist in real-world software projects. So we need some concurrent code refactoring tools to give advice of what code need change and perform the transformations automatically. It is a chance for IDE manufacturer to make the IDE more intelligent in inspecting and modifying the code. Developers will benefit a lot if such kind of automatic tools can actually help them automate their development and maintaining activities.

(3) Researchers 

\section{Related work}
\textbf{Studies on concurrent program}

\textbf{Program transformation}

\section{Conclusion}
We conduct a study on change patterns in concurrent programming.


% An example of a floating figure using the graphicx package.
% Note that \label must occur AFTER (or within) \caption.
% For figures, \caption should occur after the \includegraphics.
% Note that IEEEtran v1.7 and later has special internal code that
% is designed to preserve the operation of \label within \caption
% even when the captionsoff option is in effect. However, because
% of issues like this, it may be the safest practice to put all your
% \label just after \caption rather than within \caption{}.
%
% Reminder: the "draftcls" or "draftclsnofoot", not "draft", class
% option should be used if it is desired that the figures are to be
% displayed while in draft mode.
%
%\begin{figure}[!t]
%\centering
%\includegraphics[width=2.5in]{myfigure}
% where an .eps filename suffix will be assumed under latex, 
% and a .pdf suffix will be assumed for pdflatex; or what has been declared
% via \DeclareGraphicsExtensions.
%\caption{Simulation results for the network.}
%\label{fig_sim}
%\end{figure}

% Note that the IEEE typically puts floats only at the top, even when this
% results in a large percentage of a column being occupied by floats.


% An example of a double column floating figure using two subfigures.
% (The subfig.sty package must be loaded for this to work.)
% The subfigure \label commands are set within each subfloat command,
% and the \label for the overall figure must come after \caption.
% \hfil is used as a separator to get equal spacing.
% Watch out that the combined width of all the subfigures on a 
% line do not exceed the text width or a line break will occur.
%
%\begin{figure*}[!t]
%\centering
%\subfloat[Case I]{\includegraphics[width=2.5in]{box}%
%\label{fig_first_case}}
%\hfil
%\subfloat[Case II]{\includegraphics[width=2.5in]{box}%
%\label{fig_second_case}}
%\caption{Simulation results for the network.}
%\label{fig_sim}
%\end{figure*}
%
% Note that often IEEE papers with subfigures do not employ subfigure
% captions (using the optional argument to \subfloat[]), but instead will
% reference/describe all of them (a), (b), etc., within the main caption.
% Be aware that for subfig.sty to generate the (a), (b), etc., subfigure
% labels, the optional argument to \subfloat must be present. If a
% subcaption is not desired, just leave its contents blank,
% e.g., \subfloat[].


% An example of a floating table. Note that, for IEEE style tables, the
% \caption command should come BEFORE the table and, given that table
% captions serve much like titles, are usually capitalized except for words
% such as a, an, and, as, at, but, by, for, in, nor, of, on, or, the, to
% and up, which are usually not capitalized unless they are the first or
% last word of the caption. Table text will default to \footnotesize as
% the IEEE normally uses this smaller font for tables.
% The \label must come after \caption as always.
%
%\begin{table}[!t]
%% increase table row spacing, adjust to taste
%\renewcommand{\arraystretch}{1.3}
% if using array.sty, it might be a good idea to tweak the value of
% \extrarowheight as needed to properly center the text within the cells
%\caption{An Example of a Table}
%\label{table_example}
%\centering
%% Some packages, such as MDW tools, offer better commands for making tables
%% than the plain LaTeX2e tabular which is used here.
%\begin{tabular}{|c||c|}
%\hline
%One & Two\\
%\hline
%Three & Four\\
%\hline
%\end{tabular}
%\end{table}


% Note that the IEEE does not put floats in the very first column
% - or typically anywhere on the first page for that matter. Also,
% in-text middle ("here") positioning is typically not used, but it
% is allowed and encouraged for Computer Society conferences (but
% not Computer Society journals). Most IEEE journals/conferences use
% top floats exclusively. 
% Note that, LaTeX2e, unlike IEEE journals/conferences, places
% footnotes above bottom floats. This can be corrected via the
% \fnbelowfloat command of the stfloats package.




% conference papers do not normally have an appendix


% use section* for acknowledgment
\section*{Acknowledgment}


The authors would like to thank...





% trigger a \newpage just before the given reference
% number - used to balance the columns on the last page
% adjust value as needed - may need to be readjusted if
% the document is modified later
%\IEEEtriggeratref{8}
% The "triggered" command can be changed if desired:
%\IEEEtriggercmd{\enlargethispage{-5in}}

% references section

% can use a bibliography generated by BibTeX as a .bbl file
% BibTeX documentation can be easily obtained at:
% http://mirror.ctan.org/biblio/bibtex/contrib/doc/
% The IEEEtran BibTeX style support page is at:
% http://www.michaelshell.org/tex/ieeetran/bibtex/
\bibliographystyle{IEEEtran}
% argument is your BibTeX string definitions and bibliography database(s)
\bibliography{yfy.bib}
%
% <OR> manually copy in the resultant .bbl file
% set second argument of \begin to the number of references
% (used to reserve space for the reference number labels box)
%\begin{thebibliography}{1}
%
%\bibitem{IEEEhowto:kopka}
%H.~Kopka and P.~W. Daly, \emph{A Guide to \LaTeX}, 3rd~ed.\hskip 1em plus
%  0.5em minus 0.4em\relax Harlow, England: Addison-Wesley, 1999.
%
%\end{thebibliography}




% that's all folks
\end{document}


