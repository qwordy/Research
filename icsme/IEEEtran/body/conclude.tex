\section{Conclusion}
\label{sec:conclude}
Concurrent programming is challenging, and a mistake can introduce hidden bugs that are difficult to be detected. During software maintenance, programmers have to handle concurrent code carefully. Researchers have conducted various empirical studies to understand concurrent programming. However, how programmers maintain concurrent code is still rarely studied. In this paper, we conduct an empirical study to understand the change patterns and other perspectives of concurrent programming. Based on our analysis results, we summarize five change patterns. We show that such change patterns are repetitive in future maintenance, and programmers have confirmed the usefulness of our extracted patterns. Furthermore, our study comes to other findings such as the trends of parallel API classes and the correlation between total commits and concurrency-related commits, whose usefulness is worthy of further exploration in future work.
