\section{Related work}
\textbf{Studies on concurrent programming}
Concurrent programming attract many researchers' attention. Pinto \textit{et al}. \cite{journals/jss/PintoTFFB15} studied the usage of Java's concurrency constructs from 2227 real world, Java projects. They found that more than 75\% of projects employ concurrency control or create threads. Similarly, Wu \textit{et al}. \cite{journals/infsof/WuCZX16} conducted a study on C++ concurrency constructs. Okur and Dig \cite{conf/sigsoft/OkurD12} studied how developers use parallel libraries. They analyzed 655 open-source applications developed by 1609 programmers. The applications adopted Task Parallel Library and Parallel Language Integrated Query, which are Microsoft's parallel libraries in C\#. They reveal some interesting facts such as at least 10\% of programmers misuse the two libraries so that the programs run sequentially rather than concurrently. David \textit{et al}. \cite{conf/sosp/DavidGT13} investigated questions people wanted to know about synchronization from hardware to high-level software. They found that scalability of synchronization is a main property of hardware. Pinto \textit{et al}. \cite{conf/oopsla/PintoTC15} conducted a study on the most popular questions of concurrent programming from StackOverflow. They found that most of questions asked about concurrent programming are basic concepts such as ``what is a mutex?''. Sadowski \textit{et al}. \cite{conf/msr/SadowskiYK12} studied data races evolution from history of two open-source projects. They found that many data races exist in most time of the projects' history. Xin \textit{et al}. \cite{conf/icsm/XinQHXZWG13} conducted an empirical study on lock usage including lock manifestation and lock usage. They found that most functions which use locks only acquire one lock. Lu \textit{et al}. \cite{conf/asplos/LuPSZ08} studied characteristics of real world concurrency bugs. They had some findings such as one third of the non-deadlock concurrency bugs involve more than one variable and current detection tools cannot handle them well. Zhang \textit{et al}. \cite{journals/tse/ZhangWLQRZ16} presented a lightweight system to detect and tolerate concurrency bugs. It can detect a large range of concurrency bugs and the overhead is negligible.

\textbf{Program transformation}
Many studies have been conducted to understand program transformation and to transform programs automatically. Okur \textit{et al}. \cite{conf/ecoop/OkurED14} studied 880 open-source C\# projects and found that converting code from low-level to high-leveler parallel abstractions is tedious and error-prone. They presented two refactoring tools to help developers with the migration. Lin \textit{et al}. \cite{conf/sigsoft/LinRD14} studied 104 open-source applications to understand how AsyncTask is used, underused and misused in practice. They found that hundreds of places have the chances to apply AsyncTask and nearly half of the refactoring is done manually. They presented Asynchronizer, a refactoring tool which can extract code into AsyncTask automatically. Meng \textit{et al}. \cite{conf/pldi/MengKM11} presented Sydit, a tranformation tool which can tranform programs from examples. Sydit can make the similar edits to what human make on 70\% (39/56) cases.

\textbf{Identification of commits}