\section{Related work}
\noindent
\textbf{Empirical studies on concurrent programming.} In literature, researchers have conducted various empirical studies to understand concurrent programming. \cite{journals/jss/PintoTFFB15} conducted a large scale study on the usage of concurrency in Java, and Wu \textit{et al}. \cite{journals/infsof/WuCZX16} replicated their study with C++. Okur and Dig \cite{conf/sigsoft/OkurD12} studied how developers use parallel libraries in C\#, and their results reveal some interesting findings. For example, at least 10\% of programmers misuse libraries, so their code runs sequentially rather than concurrently. David \textit{et al}. \cite{conf/sosp/DavidGT13} conducted an empirical study to investigate synchronization at both hardware and software levels, and their results show that the scalability of synchronization is mainly a property of hardware. Pinto \textit{et al}. \cite{conf/oopsla/PintoTC15} analyze concurrency-related threads of StackOverflow, and they found that most threads discussed basic questions (\emph{e.g.}, ``what is a mutex?''). Sadowski \textit{et al}. \cite{conf/msr/SadowskiYK12} studied the evolution of data races, and they found that many data races always exist. Xin \textit{et al}. \cite{conf/icsm/XinQHXZWG13} conducted an empirical study on lock usage, and they found that most functions acquire only a lock. Lu \textit{et al}. \cite{conf/asplos/LuPSZ08} studied characteristics of real world concurrency bugs. The above approaches do not analyze change patterns and the trends of parallel APIs, which is complemented by our study. 

\noindent
\textbf{Program transformation.} Researchers proposed approaches that transform a source program to its target program. Okur \textit{et al}. \cite{conf/ecoop/OkurED14} proposed an approach that updates concurrency code with more recent APIs. For Android applications, Lin \textit{et al}. \cite{conf/sigsoft/LinRD14} proposed an approach that transforms self-written concurrency code to the Android concurrency API, \CodeIn{AsyncTask}. Meng \textit{et al}. \cite{conf/pldi/MengKM11} proposed an approach that systematically updates code according to given examples. With our findings, it is feasible to implement a tool that reduce the effort of maintaining concurrency code. 

\textbf{Identification of commits}
