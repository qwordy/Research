\section{Introduction}
Many practitioners and researchers believe that the software maintenance phase is one of the most expensive phases, in the life cycle of a software system. Some reports (\emph{e.g.}, \cite{ahn2003software}) claim that the software maintenance phase accounts for almost 80\% of the whole budget. With the maintenance of software, many revision histories are accumulated~\cite{conf/icsm/Borges16}. Based on such revision histories, researchers have conducted various empirical studies to understand how programmers maintain code (\emph{e.g.}, evolution of design patterns~\cite{aversano2007empirical}, fine-grained modifications~\cite{german2006empirical}, and the evolution of APIs~\cite{mcdonnell2013empirical}). These empirical studies deepen our understanding on software maintenance, and provide valuable insights on how to maintain code for programmer.



In recent years, to fully leverage the potential of multi-core CPUs, concurrency programming becomes increasingly popular~\cite{journals/jss/PintoTFFB15}. For example, Pinto \emph{et al.}~\cite{journals/jss/PintoTFFB15} investigated 2,227 projects, and their results show that more than 75\% of their investigated projects employ some concurrency control mechanism. Despite of its popularity, many programmers find that concurrent programming is difficult \cite{journals/corr/McKenney17}, and often introduce relevant bugs in their code~\cite{conf/asplos/LuPSZ08}. As it is difficult to maintain concurrency code, there is a strong need for a thorough empirical study on how programmers maintain such code. Despite of its importance, the topic is still rarely explored. To the best of our knowledge, only a recent study~\cite{conf/sigsoft/GuJSZL15} was conducted to understand how programmers maintain concurrency code. Although the study is insightful and explores many aspects of concurrency programming, it is still incomplete. For example, the study mainly focuses on code changes within critical sections, but maintaining concurrency code involves more code elements outside critical sections. As a result, many relevant questions are still open. For example, are there any patterns, when programmers maintain concurrency code? Indeed, such patterns are useful for programmers when they maintain code. For example, Santos \emph{et al.}~\cite{conf/icsm/SantosAEDV15} have explored the transformation patterns during software maintenance. Their results show that extracted transformation patterns can apply in new code locations. However, their study does not touch the transformation patterns of concurrency code. A more detailed analysis on how programmers maintain concurrency code can have the following benefits:




\noindent
\textbf{Benefit 1.} The results can deepen the knowledge on how to correctly maintain concurrency code. Due to the complexity of concurrency programming, we find that even experienced developers can be confused when they maintain relevant code. For example, Mark Thomas is a member of the Apache Tomcat Project Management Committee\footnote{http://tomcat.apache.org/whoweare.html}, and senior software engineer at the Covalent division of SpringSource\footnote{https://sourceforge.net/projects/covalent/}. He contributed more than 10,000 commits to Tomcat. In a commit message, he left the complain as follow:

\lstset{numbers=left, breaklines=true,  basicstyle=\ttfamily\tiny,  xleftmargin=3em, tabsize=2}
%keywordstyle=\color{blue}\bf\ttfamily, language=java,
\begin{lstlisting}
commit a6092d771ec50cf9aa434c75455b842f3ac6c628
Threading / initialisation issues. Not all were valid. Make them volatile anyway so FindBugs doesn't complain.
\end{lstlisting}

\noindent
In this example, we find that even experienced programmers can have problems in understanding their own code changes, when they maintain concurrency code. Our results can reduce such confusions.
%Using existing libraries allows you to  write less code to finish the same work and enjoy the high quality of implementation which is always reliable, strong and fast.


\noindent
\textbf{Benefit 2.} The results can be useful to improve existing tools. For example, Meng \emph{et al.}~\cite{conf/pldi/MengKM11} propose an approach that applies changes systematically based on a given example. With extensions, it can be feasible to apply our extract change patterns to update concurrency code. As another example, our results can be useful to improve existing bug detection approaches \cite{conf/ppopp/SamakR14, conf/sigsoft/EslamimehrP14}.

However, to fulfil the above benefits, we have to overcome the following challenges:

\noindent
\textbf{Challenge 1.} To ensure the reliability of our result, we have to collect many code changes that are related to concurrency programming. It is tedious to manually collect many related code changes for analysis. Tian \emph{et al.}~\cite{tian2012identifying} work on a similar research problem. They propose an approach that identifies bug fixes from commits. Their results show that even advanced techniques can fail to identify many desirable commits.


\noindent
\textbf{Challenge 2.} The changes on concurrency code can be complicated. A recent study~\cite{tufano2016there} show that only 38\% commits are compilable. To analyze code that is not compilable, researchers typically use partial-code tools such as PPA~\cite{DagenaisH08ppa} and ChangeDistiller~\cite{fluri2007change} to analyze commits. However, as partial programs lose information, partial-code tools are imprecise and typically do not support advanced analysis. Furthermore, as we do not know what patterns can be followed, it is difficult to implement a automatic tool. As a result, it is inevitable to take much human effort when we conduct the empirical study.

In this paper, we conducted an empirical study on 102,028 commits that were collected 7 popular and representative open-source projects. In our study, we extracted 696 concurrency-related commits from our collected commits, and carefully analyzed our extracted commits. Based on our results, this paper makes the following contributions:

\begin{itemize}
	\item We find some interesting findings: Some changes are contrary. Different developers may modify their code in an opposite direction. Developers are using some code-checking tools like findbugs to help them inspect their code, but sometimes these tools are not enough.
	\item We affirm that our change patterns can be applied to appropriate  contexts from real world open-source projects.
	\item We give some inspirations to concurrent program or library developers and analysis tool developers. Automated tools can be improved to help developers with program transformations.
\end{itemize}

The rest of paper is organized as follows: Section 2 presents the methodology of our study. Section 3 presents our result and discussion. Section 4 presents related work. Section 5 presents future work and Section 6 concludes.\input
