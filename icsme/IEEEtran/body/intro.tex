\section{Introduction}
\label{sec:intro}
Many practitioners and researchers believe that the software maintenance phase is one of the most expensive phases, in the life cycle of a software system. Some reports (\emph{e.g.}, \cite{ahn2003software}) claim that the software maintenance phase accounts for almost 80\% of the whole budget. With the maintenance of software, many revision histories are accumulated~\cite{conf/icsm/Borges16}. Based on such revision histories, researchers have conducted various empirical studies to understand how programmers maintain code (\emph{e.g.}, evolution of design patterns~\cite{aversano2007empirical}, fine-grained modifications~\cite{german2006empirical}, and the evolution of APIs~\cite{mcdonnell2013empirical}). These empirical studies deepen our understanding on software maintenance, and provide valuable insights on how to maintain code for programmer.

In recent years, to fully leverage the potential of multi-core CPUs, concurrent programming becomes increasingly popular~\cite{journals/jss/PintoTFFB15}. For example, Pinto \emph{et al.}~\cite{journals/jss/PintoTFFB15} investigated 2,227 projects, and their results show that more than 75\% of their investigated projects employ some concurrency control mechanism. Despite of its popularity, many programmers find that concurrent programming is difficult \cite{journals/corr/McKenney17}, and often introduce relevant bugs in their code~\cite{conf/asplos/LuPSZ08}. As it is difficult to maintain concurrent code, there is a strong need for a thorough empirical study on how programmers maintain such code. Despite of its importance, the topic is still rarely explored. To the best of our knowledge, only a recent study~\cite{conf/sigsoft/GuJSZL15} was conducted to understand how programmers maintain concurrent code. Although the study is insightful and explores many aspects of concurrent programming, it is still incomplete. For example, their study sampled only 25 concurrency-related commits. As another example, they did not analyze the change patterns of maintaining concurrent code. As a result, many relevant questions are still open. For example, are there any patterns, when programmers maintain concurrent code? Indeed, such patterns are useful for programmers when they maintain code. For example, Santos \emph{et al.}~\cite{conf/icsm/SantosAEDV15} have explored the change patterns during software maintenance. Their results show that extracted change patterns can apply in new code locations. However, their study does not touch the change patterns of concurrent code. A more detailed analysis on how programmers maintain concurrent code can have the following benefits:





\noindent
\textbf{Benefit 1.} The results can deepen the knowledge on how to correctly maintain concurrent code. Due to the complexity of concurrent programming, we find that even experienced developers can be confused when they maintain relevant code. For example, Mark Thomas is a member of the Apache Tomcat Project Management Committee\footnote{\url{http://tomcat.apache.org/whoweare.html}}, and senior software engineer at the Covalent division of SpringSource\footnote{\url{https://sourceforge.net/projects/covalent/}}. He contributed more than 10,000 commits to Tomcat. In a commit message, he left the complain as follow:

\lstset{numbers=left, breaklines=true,  basicstyle=\ttfamily\tiny,  xleftmargin=3em, tabsize=2}
%keywordstyle=\color{blue}\bf\ttfamily, language=java,
\begin{lstlisting}
commit a6092d771ec50cf9aa434c75455b842f3ac6c628
Threading / initialisation issues. Not all were valid. Make them volatile anyway so FindBugs doesn't complain.
\end{lstlisting}

\noindent
In this example, we find that even experienced programmers can have problems in understanding their own code changes, when they maintain concurrent code. Our results can reduce such confusions.
%Using existing libraries allows you to  write less code to finish the same work and enjoy the high quality of implementation which is always reliable, strong and fast.


\noindent
\textbf{Benefit 2.} The results can be useful to improve existing tools. For example, Meng \emph{et al.}~\cite{conf/pldi/MengKM11} proposed an approach that applies changes systematically based on a given example. With extensions, it can be feasible to apply our extract change patterns to update concurrent code. As another example, our results can be useful to improve existing bug detection approaches \cite{conf/sigsoft/EslamimehrP14}.

However, to fulfil the above benefits, we have to overcome the following challenges:

\noindent
\textbf{Challenge 1.} To ensure the reliability of our result, we have to collect many code changes that are related to concurrent programming. It is tedious to manually collect many related code changes for analysis. Tian \emph{et al.}~\cite{tian2012identifying} worked on a similar research problem. They proposed an approach that identifies bug fixes from commits. Their results show that even advanced techniques can fail to identify many desirable commits.


\noindent
\textbf{Challenge 2.} The changes on concurrent code can be complicated. A recent study~\cite{tufano2016there} showed that only 38\% commits are compilable. To analyze code that is not compilable, researchers typically use partial-code tools such as PPA~\cite{DagenaisH08ppa} and ChangeDistiller~\cite{fluri2007change} to analyze commits. However, as partial programs lose information, partial-code tools are imprecise and typically do not support advanced analysis. Furthermore, as we do not know what patterns can be followed, it is difficult to implement an automatic tool. As a result, it is inevitable to take much human effort when we conduct the empirical study.

In this paper, we conduct an empirical study on 98,352 commits that are collected from six popular and representative open-source projects. To reduce the effort of manual inspection, we implement a set of tools that collect and identify concurrency-related commits automatically (see Section~\ref{sec:method:tool} for details). With its support, in total, we identified 695 concurrency-related commits, and conducted careful analysis on our identified commits. Based on our results, this paper makes the following contributions:

\begin{itemize}
	\item The first analysis on the change patterns of maintaining concurrent code. Based on our results, we summarize five change patterns, and we present examples to explain each type of our change patterns. We find that following such change patterns, during software maintenance, programmers can modify concurrent code to repair bugs, improve performance, and change functions of their code. Furthermore, we find that maintaining concurrent code is not a one-direction migration. Due to various considerations, programmers can apply seemingly contradictory changes, and even revert their changes. Sometimes, programmers can even make changes, before they fully understand the consequences of their changes.
	\item An application of our change patterns in real code. In particular, we search the latest versions of three projects for chances to apply our change patterns, and synthesize three pull requests according to our change patterns. One of our pull requests is already confirmed and accepted by their programmers. However, our results also reveal that it needs much experience and understanding to leverage our change patterns.
	\item More finding on concurrent program. For example, we reveal the trends of different parallel API classes. As another example, we find that at the granularity of commits, there still exists a correlation between total commits and concurrency-related commits, but it is less significant than at the granularity of code lines.
\end{itemize}

The rest of paper is organized as follows: Section~\ref{sec:method} presents the methodology of our study. Section~\ref{sec:result} presents our results. Section~\ref{sec:discuss} discusses our future work. Section~\ref{sec:related} presents related work. Section~\ref{sec:conclude} concludes this paper.
